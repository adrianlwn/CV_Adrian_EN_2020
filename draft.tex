%%%%%%%%%%%%%%%%%%%%% PROJECTS  %%%%%%%%%%%%%%%%%%%%% 

\section{Data Science Projects}

\subsection{Gaussian Processes for Optimal Sensor Position}
\location{Imperial College, London | Summer 2019}
\descript{Master Thesis}
\longdescript{Employment of a \textbf{Gaussian Process} model to calculate the optimal spatial positioning of sensors to study and collect air pollution data in big cities. Validation with Data Assimilation. Big Data Problem. Imperial College London \textbf{Data Science Institute}. 
}
\sectionsep



\subsection{NLP Challenge - SemEval 2019 Task 6}
\location{Imperial College, London | Spring 2019}
\descript{Codalab Competition | \href{https://github.com/adrianlwn/SemEval-2019-Task-6}{\faGithubSquare{} Github Repository}}
\longdescript{Classification of Offensive Tweets. Globally \textbf{obtained the 5th best results}. Use of state of the art methods such as GRU, LSTM, RNN or CNN. }
\sectionsep

\subsection{Tweet Awareness - Data Analysis}
\location{EPFL, Lausanne | Autumn 2017}
\descript{Group Project | \href{https://tweet-awareness.eu}{\faGlobe{} Data Story}}
\longdescript{Data Analysis Project on how to measure the \textbf{awareness} of people about dramatic events around the world and how to correlate it to \textbf{cultural distances}. Data \textbf{extraction} from twitter using Python (Selenium, BeautifulSoup). Data \textbf{analysis} using Python (Pandas, Sklearn). Data \textbf{visualisation} using Javascript (D3.js).
Group Project created in the context of the Applied Data Analysis course of Pr. R. West.}
\sectionsep

%\subsection{Speech Recognition Challenge - Network Data Science}
%\location{EPFL, Lausanne | Autumn 2017}
%\descript{Kaggle Competition |  %\href{https://github.com/adrianlwn/NTDS-Speech-Recognition-Challenge
%}{\faGithubSquare{} Github Repository}}
%\longdescript{Network Tour of Data Science Project. Classifying noisy audio commands. Cleaning and cutting of Audio Signals. Feature Extraction using mel-cepstral cepstrum. Building of a graph and extraction the fiedler vectors from the Laplacian to cluster the graph and classify the audio signals.  }
%\sectionsep

\section{Smart Grid Projects}

\subsection{Provision of Multiple Services to the Grid with Plug-In-Electrical-Vehicles}
\location{EPFL, Lausanne | Spring 2018}
\descript{Master Thesis |  \href{https://stisrv13.epfl.ch/masterposter/960.pdf}{\faBook{} Poster of the Project} }
\longdescript{Using Electrical Vehicles for providing services to the grid, such as Frequency Regulation. Optimisation problem using real transportation data to determine the regulation capacity for the electricity markets. Using Matlab, Gurobi solver and YALMIP. Supervised by Pr. C. Jones.}
\sectionsep

\subsection{Robust restoration in DG-incorporated distribution networks}
\location{EPFL, Lausanne | Autumn 2017}
\descript{MSc Semester Project }
\longdescript{Formulation and implementation of the \textbf{Restoration Problem} in Electrical Grid, a Mixed-Integer-Non-Linear Optimisation Problem. Implementation using Matlab and the Gurobi solver. Supervised by Dr. R. Cherkaoui.}
\sectionsep

\subsection{ETR applied to Fault Detection in Power Networks}
\location{EPFL, Lausanne | Spring 2017}
\descript{MSc Semester Project | \href{https://infoscience.epfl.ch/record/256592?ln=en}{\faBook{} Conference Paper}}
\longdescript{Investigating the physical application of the Electromagnetic Time Reversal (ETR) principle, in the context of fault detection in \textbf{Electrical Grids}.  Supervised by Pr. F. Rachidi.  Three weeks residency spend at \textbf{Amir-Kabir University in Tehran} for this project. }
\sectionsep

\subsection{H2O2 Fuel Cell and Electrolyser Analysis and Monitoring}
\location{EPFL, Lausanne | Spring 2016 }
\descript{BSc Project | \href{https://desl-pwrs.epfl.ch}{\faGlobe{} EPFL Microgrid}}
\longdescript{Implementation of a \textbf{Monitoring System} (in LabView) for a Fuel Cell and Electrolyser in the context of a Lab Microgrid. Supervised by Pr. M. Paolone.}
\sectionsep
