\documentclass{article}
% Package Imports

\usepackage[a4paper, hmargin=1cm, vmargin=1.5cm]{geometry}
\usepackage[hidelinks]{hyperref}
\usepackage[absolute]{textpos}
\usepackage[UKenglish]{babel}
\usepackage[UKenglish]{isodate}
\usepackage[dvipsnames]{xcolor}
\usepackage{fontspec,xltxtra,xunicode}
\usepackage{titlesec}
\usepackage{fontawesome}

\pagestyle{empty}


% Color definitions
\definecolor{darkgray}{HTML}{555555} 
\definecolor{primary}{HTML}{000000} 
\definecolor{headings}{HTML}{6A6A6A}
\definecolor{subheadings}{HTML}{555555}

% Set main fonts
\defaultfontfeatures{Mapping=tex-text}
\setmainfont{Lato}[
    Path = fonts/,
    Extension = .ttf,
    UprightFont = *-Regular,
    BoldFont = *-Bold,
    ItalicFont= *-Italic
 ]


\newfontfamily{\headingfont}{Raleway}[
    Path = fonts/,
    Extension = .ttf,
    UprightFont = *-Regular,
    BoldFont = *-SemiBold
    ]

% Headings command
\titleformat{\section}{\color{headings}\headingfont\scshape\bfseries\LARGE}{}{0em}{}[{\titlerule[0.4pt]}]
\titlespacing{\section}{0pt}{3pt}{5pt}

% Subsection command
\titleformat{\subsection}[runin]{\color{primary}\headingfont\bfseries\large}{}{0em}{}[]
\titlespacing*{\subsection}{0pt}{0pt}{8pt}

% Description COmmand 
\newcommand{\descript}[1]{\color{darkgray}\normalfont \textbf{#1\\}}

\newcommand{\longdescript}[1]{\color{subheadings}\normalfont\small {#1\\} }

% Location command
\newcommand{\location}[1]{\color{primary}\headingfont {\hfill #1\\} }

% Section seperators command
\newcommand{\sectionsep}[0]{\vspace{-6pt}}
% Bullet Lists with fewer gaps command
\newenvironment{tightemize}{\vspace{-2\topsep}\begin{itemize}\itemsep1pt \parskip0pt \parsep0pt}{\end{itemize}\vspace{\topsep}}

% Name command
\newcommand{\namesection}[4]{
	\begin{center}
		\sffamily
		\headingfont\fontsize{35pt}{14pt}\selectfont\scshape #1 
			\headingfont\selectfont\scshape\bfseries #2
	\end{center}
	\vspace{-14pt}
		\begin{center} \color{subheadings}\normalfont\fontsize{11pt}{14pt}\selectfont #3
		\end{center}
	
	%\noindent\makebox[\linewidth]{\rule{\paperwidth}{0.4pt}}
	\vspace{-5pt}
	}

\newcommand{\mybullet}[1]{ \hspace{2pt}\textbullet{#1}\hspace{2pt} }
%%%%%%%%%%%%%%%%%%%%% BEGIN %%%%%%%%%%%%%%%%%%%%%%%%
%%%%%%%%%%%%%%%%%%%%% DOCUMENT %%%%%%%%%%%%%%%%%%%%%

\begin{document}
%%%%%%%%%%%%%%%%%%%% NAME %%%%%%%%%%%%%%%%%%%% 

\namesection{Adrian}{Löwenstein}{\href{https://www.adrianlwn.com}{\faGlobe{} adrianlwn.com} | \href{https://github.com/adrianlwn}{\faGithubSquare{}  github}  | \href{https://www.linkedin.com/in/adrianloewenstein}{\faLinkedinSquare{}  linkedin} \\ \href{mailto:adrian.loewenstein@icloud.com}{\faEnvelope{} adrian.loewenstein@icloud.com} |  \href{tel:+33767885257}{\faPhoneSquare{} +33 7 67 88 52 57}  \\ French Nationality | Born 21.11.1993}
%{``On ne voit bien qu'avec le cœur. L'essentiel est invisible pour les yeux.''}




%%%%%%%%%%%%%%%%%%%% TECHNICAL SKILLS %%%%%%%%%%%%%%%%%%%% 

\section{Technical Skills}

\subsection{Certification} 
\longdescript{Azure Data Engineer Associate [DP-200 + DP-201]
}
\sectionsep

\subsection{Programming}
\longdescript{Python [Pytorch, Keras, Pandas, Spark, Scikit-Learn, Matplotlib, ... ] \mybullet{} C \mybullet{} C++ \mybullet{} Matlab \mybullet{} SQL\mybullet{} Linux [bash] \\ \hspace*{3pt} Devops Practices [Git, Gitlab CI/CD]  \mybullet{} Javascript [D3.js]   }
\sectionsep

\subsection{Computer Science} 
\longdescript{ Classical Machine Learning \mybullet{} Deep Learning \mybullet{} Natural Language Processing [Transformers, Spacy]  \\ \hspace*{3pt} Time Series \mybullet{} Computer Vision \mybullet{} Reinforcement Learning  \mybullet{} Probabilistic Inference [Gaussian Processes, Bayesian Opt. , VAE ]   } 
\sectionsep

\subsection{Electrical Engineering}
\longdescript{ \mybullet{} Model Predictive Control \mybullet{} Control Theory \mybullet{} Computational Optimisation \\  \hspace*{3pt}  Electrical Grid Control \& Modelling \mybullet{} Electricity Market \mybullet{} Energy Storage \mybullet{} Energy Generation}
\sectionsep




%%%%%%%%%%%%%%%%%%%%% EXPERIENCE %%%%%%%%%%%%%%%%%%%%% 
\section{Professional Experience}

\subsection{Quantmetry}
\location{Paris, France | Sept 2019 - Today}
\descript{Data Scientist}
\longdescript{Working as a Data Scientist consultant for various projects and companies. Developing solutions and interacting extensively with clients.  \\ \textbf{NLP} : Developing, for a medical news company, tools enabling the profiling and clustering of doctors by using NLP techniques and dimension reduction on the content of the media. \\ \textbf{NLP} : Creation of supporting material, practical exercises and training given to other Data Scientists on latest NLP subjects. \\ \textbf{Time Series ans Supply Chain} : Developing, for a wholesaler, of a tailored solution in supply. Including a sales prediction module at the SKU level, and a supply optimisation module. Using Deep Learning approaches and constrained optimisation methods.\\ \textbf{Data Engineering} : Developing infrastructures for the French Ministry of Health in support of the internal Covid-19 Data platform. Including Gitlab CI/CD pipelines, Docker runners, sftp servers. \\ \textbf{Computer Vision} : Developing, for a fashion startup, image segmentation models by using \textbf{Computer Vision} techniques and \textbf{Deep Learning} on Pytorch. Deployed with Django APIs on Google Cloud Platform. } 
\sectionsep

\subsection{Ecole Polytechnique Fédérale de Lausanne | EPFL}
\location{Lausannne, Switzerland | 2015 - 2018}
\descript{Teaching Assistant }
%\longdescript{Electrical Systems and Electronics I | Dr. Adil Koukab | Spring 2017 \& 2018\\
%Electrotechnics I | Pr. Oliver Martin | Autumn 2015 \& 2016 \\
%Microcontrollers | Pr. Alexandre Schmidt | Spring 2015
%}
\longdescript{Working as a Teacher Assistant for several Professors during my studies at EPFL}
\sectionsep
\subsection{Commissariat à L’Energie Atomique | CEA}
\location{Le Bourget du Lac, France | June 2016 - Aug 2016}
\descript{MSc Internship}
\longdescript{Internship at \textit{Institut National de l'Energie Solaire} (INES) - Development and validation of energy management strategies in Smart Grids. Study of battery models and implementation in a battery simulator. Evaluation on a Solar Microgrid of battery charging strategies.}
\sectionsep

\subsection{Airbus Helicopters UK }
\location{Oxford, United Kingdom |  June 2011 – July 2011}
\descript{Internship}
\longdescript{Introductory Internship done between the French Baccalaureate and the beginning of my studies.}
\sectionsep

%%%%%%%%%%%%%%%%%%%%% EDUCATION %%%%%%%%%%%%%%%%%%%%% 
\section{Education}
\subsection{Imperial College London}
\location{London, United Kingdom | Sept. 2019 }
\longdescript{MSc in Computing | \textbf{Machine Learning} | obtained with Merit }
\sectionsep
\subsection{Ecole Polytechnique Fédérale de Lausanne | EPFL }
\location{Lausanne, Switzerland | July 2018}
\longdescript{MSc in Electrical Engineering  | \textbf{Energy \& Smart Grids Science} | Average : 5.31 / 6.0}
\sectionsep

\subsection{Eidgenössische Technische Hochschule Zürich | ETHZ }
\location{Zürich, Switzerland | Aug. 2015}
\longdescript{Exchange in 3rd Year of BSc}
\sectionsep

\subsection{Ecole Polytechnique Fédérale de Lausanne | EPFL}
\location{Lausanne, Switzerland | Feb. 2016}
\longdescript{BSc in Electrical Engineering | Average : 4.8 / 6.0}
\sectionsep

%\subsection{Lycée Pasteur}
%\location{Neuilly-sur-Seine, France | July 2012}
%\longdescript{Preparatory Course for Engineering Schools}
%\sectionsep

%\subsection{Institut de la Tour}
%\location{Paris, France | July 2011}
%\descript{High-school | French Baccalaureate  - Highest Honors}
%\sectionsep

%%%%%%%%%%%%%%%%%%%% SOFT SKILLS %%%%%%%%%%%%%%%%%%%% 
\section{Soft Skills}
\subsection{Qualities}
\longdescript{Reasoning \mybullet{} Analysis \mybullet{} Adaptability \mybullet{} Curiosity \mybullet{} Creativity \mybullet{} Team Spirit }
\sectionsep
\subsection{Language}
\longdescript{French [Native] \mybullet{} English [C1] \mybullet{} German [B2] }
\sectionsep

%%%%%%%%%%%%%%%%%%%%% EXTRA CURICULAR %%%%%%%%%%%%%%%%%%%%% 
\section{Extra Curicular}
\subsection{Associative}
\longdescript{EPFL Electrical Engineering Students Association \mybullet{} EPFL Instagram Student Photographer [@epflstudents] }
\sectionsep
\subsection{Others}
\longdescript{Photography [Digital, Analog] \mybullet{}  Video Editing \mybullet{} Travel Biking \mybullet{} Hiking \mybullet{} Skiing \mybullet{} Discovery Travelling [Peru, China, Iran]} 
\sectionsep


%%%%%%%%%%%%%%%%%%%%% PROJECTS  %%%%%%%%%%%%%%%%%%%%% 

\section{Data Science Projects}

\subsection{Gaussian Processes for Optimal Sensor Position}
\location{Imperial College, London | Summer 2019}
\descript{Master Thesis}
\longdescript{Employment of a \textbf{Gaussian Process} model to calculate the optimal spatial positioning of sensors to study and collect air pollution data in big cities. Validation with Data Assimilation. Big Data Problem. Imperial College London \textbf{Data Science Institute}. 
}
\sectionsep



\subsection{NLP Challenge - SemEval 2019 Task 6}
\location{Imperial College, London | Spring 2019}
\descript{Codalab Competition | \href{https://github.com/adrianlwn/SemEval-2019-Task-6}{\faGithubSquare{} Github Repository}}
\longdescript{Classification of Offensive Tweets. Globally \textbf{obtained the 5th best results}. Use of state of the art methods such as GRU, LSTM, RNN or CNN. }
\sectionsep

\subsection{Tweet Awareness - Data Analysis}
\location{EPFL, Lausanne | Autumn 2017}
\descript{Group Project | \href{https://tweet-awareness.eu}{\faGlobe{} Data Story}}
\longdescript{Data Analysis Project on how to measure the \textbf{awareness} of people about dramatic events around the world and how to correlate it to \textbf{cultural distances}. Data \textbf{extraction} from twitter using Python (Selenium, BeautifulSoup). Data \textbf{analysis} using Python (Pandas, Sklearn). Data \textbf{visualisation} using Javascript (D3.js).
Group Project created in the context of the Applied Data Analysis course of Pr. R. West.}
\sectionsep

%\subsection{Speech Recognition Challenge - Network Data Science}
%\location{EPFL, Lausanne | Autumn 2017}
%\descript{Kaggle Competition |  %\href{https://github.com/adrianlwn/NTDS-Speech-Recognition-Challenge
%}{\faGithubSquare{} Github Repository}}
%\longdescript{Network Tour of Data Science Project. Classifying noisy audio commands. Cleaning and cutting of Audio Signals. Feature Extraction using mel-cepstral cepstrum. Building of a graph and extraction the fiedler vectors from the Laplacian to cluster the graph and classify the audio signals.  }
%\sectionsep

\section{Smart Grid Projects}

\subsection{Provision of Multiple Services to the Grid with Plug-In-Electrical-Vehicles}
\location{EPFL, Lausanne | Spring 2018}
\descript{Master Thesis |  \href{https://stisrv13.epfl.ch/masters/img/960.pdf}{\faBook{} Poster of the Project} }
\longdescript{Using Electrical Vehicles for providing services to the grid, such as Frequency Regulation. Optimisation problem using real transportation data to determine the regulation capacity for the electricity markets. Using Matlab, Gurobi solver and YALMIP. Supervised by Pr. C. Jones.}
\sectionsep

\subsection{Robust restoration in DG-incorporated distribution networks}
\location{EPFL, Lausanne | Autumn 2017}
\descript{MSc Semester Project }
\longdescript{Formulation and implementation of the \textbf{Restoration Problem} in Electrical Grid, a Mixed-Integer-Non-Linear Optimisation Problem. Implementation using Matlab and the Gurobi solver. Supervised by Dr. R. Cherkaoui.}
\sectionsep

\subsection{ETR applied to Fault Detection in Power Networks}
\location{EPFL, Lausanne | Spring 2017}
\descript{MSc Semester Project | \href{https://infoscience.epfl.ch/record/256592?ln=en}{\faBook{} Conference Paper}}
\longdescript{Investigating the physical application of the Electromagnetic Time Reversal (ETR) principle, in the context of fault detection in \textbf{Electrical Grids}.  Supervised by Pr. F. Rachidi.  Three weeks residency spend at \textbf{Amir-Kabir University in Tehran} for this project. }
\sectionsep

\subsection{H2O2 Fuel Cell and Electrolyser Analysis and Monitoring}
\location{EPFL, Lausanne | Spring 2016 }
\descript{BSc Project | \href{https://desl-pwrs.epfl.ch}{\faGlobe{} EPFL Microgrid}}
\longdescript{Implementation of a \textbf{Monitoring System} (in LabView) for a Fuel Cell and Electrolyser in the context of a Lab Microgrid. Supervised by Pr. M. Paolone.}
\sectionsep




\end{document}
